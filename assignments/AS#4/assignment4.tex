% Preamble:
\documentclass{article}

% Packages:
\usepackage{fancyhdr}
\usepackage{amsmath}

% Title Page Information:
\title{CS23 Assignment Four (Counting, pt.1)}
\author{CJ Bridgman-Ford \\ cj.ikaika@gmail.com}
\date{April 4, 2024}

% Make subsections lettered:
\renewcommand{\thesubsection}{\alph{subsection}.}

% fancyhdr Page Styling:
\newcommand{\pagenumber}{\thepage\quad}
\newcommand{\authorname}{CJ Bridgman-Ford}

\pagestyle{fancy}
\renewcommand{\headrulewidth}{0pt}
\fancyhead{}
\fancyfoot[L]{\authorname}
\fancyfoot[C]{}
\fancyfoot[R]{\pagenumber}

% End of Preamble.

% Start of document:
\begin{document}

% Title Page:
\maketitle
\thispagestyle{empty}


\clearpage
% Page 1:
\pagenumbering{arabic}

% Problem 1:
\section{For your college interview, you must wear a tie.
    You own 3 regular (boring) ties and 5 (cool) bow ties.}
\subsection{How many choices do you have for neckwear?} 
\hspace{1cm}\textit{Assuming you are wearing only one tie, that gives you $5+3$ options.
    Therefore, the answer is 8.}
\subsection{You realize that the interview is for clown college,
    so you should probably wear both a regular tie and a bow tie. How many choices do you have now?}
 \hspace{1cm}\textit{For each of the $3$ regular ties, you have the option of $5$ different bow ties
    to pick from. Therefore, the answer is $3 \times 5 = 15$.}
\subsection{For the rest of your outfit, you have 5 shirts, 4 skirts, 3 pants, and 7 dresses.
    You want to select either a shirt to wear with a skirt or pants, or just a dress.
    How many outfits do you have to choose from?}
\hspace{1cm}\textit{Let's start with just the two piece outfit. You have $5$ shirts that can be paired
    with a skirt or dress. Between the $4$ skirts and $3$ pants, that gives you $5\times(4+3)$ options.
    However, our overall choice allows us to choose between a one or a two piece outfit. This gives us a
    total of $5\times(4+3)+7 = 42$ different outfits.}
\clearpage

% Problem 2:
\section{Hexadecimal, or base 16, uses 16 distinct digits that can be used to form numbers:
\\ \{0,1,...,9,A,B,C,D,E,F\}. So for example, \\ a 3 digit hexadecimal number might be 2B8.}
\subsection{How many 2-digit hexadecimals are there in which the first digit is E or F?
    Explain your answer in terms of the additive principle (using either events or sets).}
\hspace{1cm}\textit{The additive property tells us that if $A$ occurs $m$ way, and $B$
    occurs in $n$ disjoint ways, then the event $A$ or $B$ can occur in in $m+n$ different ways.
    For both $E$ and $F$, there are $16$ different ways, respectively. Therefore, the total
possibilities is equal to $16+16=32$.}
\subsection{Explain why your answer to the previous part is correct in terms of the
    multiplicative principle (using either events or sets).
    Why do both the additive and multiplicative principles give you the same answer?}
\hspace{1cm}\textit{The multiplicative principle states that if $A$ can happen in $m$ ways and $B$
    (which depends on the outcome of $A$) can happen in $n$ ways, the total possible outcomes is
    $m\times n$. There are two possibilities for $A$ (either E or F), and 16 possibilities for $B$. Therefore,
    the total solutions is $2\times 16 = 32$. The answers between parts a and b are the same because
    you are fundamentally calculating the same event, with two different perspectives on the
    relationship between $A$ and $B$.}
\subsection{How many 3-digit hexadecimals start with a letter (A-F) and end with a numeral (0-9)? Explain.}
\hspace{1cm}\textit{Let's call the digits $A,B,C$ from left to right. Therefore, $A$ has $6$ possibilities (A-F),
    $B$ has $16$ (0-F), and $C$ has $10$ (0-9). We can now use the multiplicative property: 
    $6\times 16\times 10 = 960$ possibilities.}
\clearpage
\subsection{How many 3-digit hexadecimals start with a letter (A-F) or end with a numeral (0-9) (or both)? Explain.}
\hspace{1cm}\textit{This question presents us with three cases. We can use our logic from part c to
    analyze them:}
\begin{center}
    \begin{tabular}{c|c|c|c|c}
        Case & A & B & C & Outcomes \\
        \hline
        1 & 6 & 16 & 16 & 1536 \\
        \hline
        2 & 6 & 16 & 10 & 940 \\
        \hline
        3 & 16 & 16 & 10 & 2560 \\
        \hline
    \end{tabular}
\end{center}
\hspace{1cm}\textit{By adding cases $1$ and $3$ while subtracting $2$, we get the total unique
    possibilities: $1536+2560-940 = 3156$.}

% Problem 3:
\section{If $|M| = 100$ and $|N| = 42$, what is $|M\cup N|+|M\cap N|$?}
\hspace{1cm}\textit{Using the formula $|M\cup N|=|M|+|N|-|M\cap N|$ we can isolate our knowns and unknowns
    on each side of the equation. This gives us $|M\cup N|+|M\cap N| = |M|+|N|$. So,$|M\cup N|+|M\cap N|$
    is equivalent to $100+42 = 142$.}

% Problem 4:
\section{Consider all 5 letter “words” made from the letters a through f.
    (Recall, words are just strings of letters, not necessarily actual English words.)}
\subsection{How many of these words are there total?}
\hspace{1cm}\textit{For each letter, there are $6$ possibilities. Therefore, we can find the total number
    of possibilities like so: $6\times 6\times 6\times 6\times 6 = 6^5 = 7776$.}
\subsection{How many of these words contain no repeated letters?}
\hspace{1cm}\textit{We can use a modified logic from part a, by decreasing the options in each subsequent event:
    $6\times 5\times 4\times 3\times 2 = 6! = 720$.}
\subsection{How many of these words start with the sub-word \\ “aba”?}
\hspace{1cm}\textit{Since the first three letters are already determined to be aba, that leaves us with
    $6\times 6 = 36$ possibilities.}
\clearpage
\subsection{How many of these words either start with “abc” or end with “cba” or both?}
\hspace{1cm}\textit{We can use a modified logic from problem 2c to analyze this problem:}
\begin{center}
    \begin{tabular}{c|c|c|c}
        Case & A & B & Outcomes \\
        \hline
        abc--- & 6 & 6 & 36 \\
        \hline
        ---cba & 6 & 6 & 36 \\
        \hline
        both: abcba &  &  & 1 \\
        \hline
    \end{tabular}
\end{center}
\hspace{1cm}\textit{Using the same logic as in problem 2c, we perform the operation $36+36-1$,
    which tells us that there are $71$ unique possibilities.}
\subsection{How many of the words containing no repeats also do not contain the sub-word “bad”?}
\hspace{1cm}\textit{We can, again, use a modified logic from problem 2c in tandem with logic from
    3b to analyze this problem: (Note: A and B will be 3 and 2, respectively, because there are no
    repeats)}
\begin{center}
     \begin{tabular}{c|c|c|c}
         Case & A & B & Outcomes \\
         \hline
         ---bad & 3 & 2 & 6 \\
         \hline
         --bad-- & 3 & 2 & 6 \\
         \hline
         bad--- & 3 & 2 & 6 \\
         \hline
     \end{tabular}
 \end{center}
 \hspace{1cm}\textit{Adding up the cases gives us all of the possibilities with "bad". If we look back
    on part b, we see that there were 720 unique possibilities without repeats. Thus, this will give us
    the number of words containing no repeats also do not contain the sub-word “bad”:
    $720-(6+6+6) = 702$.}

% Problem 5:
\section{Consider the bit strings in B(6,2) (bit strings of length 6 and weight 2).}
\subsection{How many of those bit strings start with 1?}
\hspace{1cm}\textit{Since the first slot is already given to be 1, that leaves us with 5 remaining slots to 
    fill with a single 1. We can use $C(5,1) = \frac{5!}{1!(5-1)!}$ to calculate the possibilities.
    This yields $5$ possibilities.}
\subsection{How many of those bit strings start with 01?}
\hspace{1cm}\textit{Using the same logic from part a, we use the calcuation:
    $C(4,1) = \frac{4!}{1!(4-1)!} = 4$ possibilities.}
    \clearpage
\subsection{How many of those bit strings start with 001?}
\hspace{1cm}\textit{Using the same logic from part a, we use the calcuation:
        $C(3,1) = \frac{3!}{1!(3-1)!} = 3$ possibilities.}
\subsection{Are there any other strings we have not counted yet? Which ones, and how many are there?}
\hspace{1cm}\textit{We have yet to count $C(2,1)=2$ and $C(1,1)=1$. That is a total of 3 strings.}
\subsection{How many bit strings are there total in B(6,2)}
\hspace{1cm}\textit{We have six spots to fill with two $1's$. Therefore, the equation
    $C(6,2)$ will yield our result of $15$ total possibilities}

\end{document}