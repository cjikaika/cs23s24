% Preamble:
\documentclass{article}

% Packages:
\usepackage{fancyhdr}
\usepackage{amsmath}

% Title Page Information:
\title{CS23 Assignment Three}
\author{CJ Bridgman-Ford \\ cj.ikaika@gmail.com}
\date{April 4, 2024}

% Make subsections lettered:
\renewcommand{\thesubsection}{\alph{subsection}.}

% fancyhdr Page Styling:
\newcommand{\pagenumber}{\thepage\quad}
\newcommand{\authorname}{CJ Bridgman-Ford}

\pagestyle{fancy}
\renewcommand{\headrulewidth}{0pt}
\fancyhead{}
\fancyfoot[L]{\authorname}
\fancyfoot[C]{}
\fancyfoot[R]{\pagenumber}

% End of Preamble.

% Start of document:
\begin{document}

% Title Page:
\maketitle
\thispagestyle{empty}


\clearpage
% Page 1:
\pagenumbering{arabic}

% Problem 1:
\section{Is $(1, 2, 3, 4) = (1, 2, 4, 3)$?}
\hspace{1cm}\textbf{No.}\textit{ Unlike sets, the order of elements in a tuple matters.}

% Problem 2
\section{Which of the following are equivalent: \\
    $\{a, b, c\}, \{\{a, b\}, c\}, (a, b, c), (a, (b, c)), (b, c, a),
    \\ \{\{a, b, c\}\}, \{b, c, a\}, \{\}, \{\{\}\}$}
\hspace{1cm}\textbf{$\{a, b, c\}$ is equivalent to $\{b, c, a\}$.}\textit{
    All other tuples and sets are inequivalent due to differences in strcture
    or order of elements.}

% Problem 3:



\end{document}