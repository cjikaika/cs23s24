% Preamble:
\documentclass{article}

% Packages:
\usepackage{fancyhdr}
\usepackage{amsmath}

% Title Page Information:
\title{CS23 Assignment Two (Statements)}
\author{CJ Bridgman-Ford \\ cj.ikaika@gmail.com}
\date{February 25, 2024}

% Make subsections lettered:
\renewcommand{\thesubsection}{\alph{subsection}.}

% fancyhdr Page Styling:
\newcommand{\pagenumber}{\thepage\quad}
\newcommand{\authorname}{CJ Bridgman-Ford}

\pagestyle{fancy}
\renewcommand{\headrulewidth}{0pt}
\fancyhead{}
\fancyfoot[L]{\authorname}
\fancyfoot[C]{}
\fancyfoot[R]{\pagenumber}

% End of Preamble.

% Start of document:
\begin{document}

% Title Page:
\maketitle
\thispagestyle{empty}


\clearpage
% Page 1:
\pagenumbering{arabic}

% Problem 1:


\section{Problem 1: Suppose $P$ and $Q$ are the statements: \\ P: Harry plays Magic: The Gathering. \\ Q: Sally plays Magic: The Gathering.}
\subsection{Translate ``Harry and Sally both play Magic: The Gathering'' into symbols.}
\hspace{1cm}\textit{$P$ and $Q$ are both true: $P \land Q$}
\subsection{Translate ``If Harry plays Magic: The Gathering, Sally does not'' into symbols.}
\hspace{1cm}\textit{$P$ is true, thus $Q$ is not, therefore: $P \xrightarrow{}\neg Q$}
\subsection{Translate ``$P\lor Q$'' into English.}
\hspace{1cm}\textit{$P$ or $Q$ is true, therefore: Harry and or Sally may play Magic: The Gathering; we know that at least one of them is a player.} 
\subsection{Translate ``$\neg (P \land Q) \xrightarrow{} Q$'' into English.}
\hspace{1cm}\textit{The opposite of ($P$ and $Q$), then $Q$. Therefore: If the statement   ``both Harry and Sally play Magic: The Gathering'' is false, then Sally plays Magic: The Gathering.}

% Problem 2:


\section{Construct the Truth Table for the following statement form: $(P \lor Q) \lor (\neg P \land Q) \xrightarrow{} Q$}
\hspace{1cm}\textit{($P$ or $Q$) or ((not $P$) and $Q$), then $Q$}
\\\\
\vspace{1cm}
\begin{tabular}{c|c|c|c|c|c}
\hline
$P$ & $Q$ & $P \lor Q$ & $\neg P \land Q$ & $(P \lor Q) \lor (\neg P \land Q)$ & $(P \lor Q) \lor (\neg P \land Q) \xrightarrow{} Q$\\
\hline
T & T & T & F & T & T \\
T & F & T & F & T & F \\
F & T & T & T & T & T \\
F & F & F & F & F & T \\
\hline
\end{tabular}

% Problem 3:


\section{Problem 3: Determine whether each statement below is true or false, or whether it is impossible to determine. \\ Assume you do not know what my favorite number is, (but you do know 13 is a prime).}
\textbf{$P$: 13 is a prime number.}\\
\textbf{$Q$: 13 is my favorite number.}\\
\textbf{$S$: 7 is my favorite number.}\\

\subsection{If 13 is prime, then 13 is my favorite number.}
\hspace{1cm}\textit{$P \xrightarrow{} Q$? $P$ and $Q$ are logically independent. Therefore, even though $P$ is true, we cannot determine whether or not the overall statement itself is true.}
\subsection{If 13 is my favorite number, then 13 is prime.}
\hspace{1cm}\textit{$Q \xrightarrow{} P$? $Q$ and $P$ are logically independent. However, $P$ is known to be true. Therefore, the statement is true.}
\subsection{If 13 is not prime, then 13 is my favorite number.}
\hspace{1cm}\textit{$\neg P \xrightarrow{} Q$. $P$ and $Q$ are logically independent. $\neg P$ is false, but we cannot determine what $Q$ is. Therefore, the statement is impossible to determine.}
\subsection{13 is my favorite number or 13 is prime.}
\hspace{1cm}\textit{$Q \lor P$? This statement is true because $P$ is known to be true. Since this is $P$ OR $Q$, it does not matter that Q is unknown.}
\subsection{13 is my favorite number and 13 is prime.}
\hspace{1cm}\textit{$Q \land P$? This statement is impossible to determine because $Q$ is unknown. The AND operation requires both inputs to be true. $P$ is known to be true but $Q$ is inconclusive.}
\subsection{7 is my favorite number and 13 is not prime.}
\hspace{1cm}\textit{$S \land \neg P$? This statement is false, as $P$ is known to be true, therefore $\neg P$ is false. The AND operation requires both inputs to be true. Therefore, if at least one input is false, the whole statement becomes false.}
\subsection{13 is my favorite number or 13 is not my favorite number.}
\hspace{1cm}\textit{$Q \lor \neg Q$. This statement is true. $Q$ can either be false or true, but must exist in one state or the other. The OR operation requires at least one input to be true. Since we are inputting both possibilities, at least one will be true, and thus the statement will be true.}

% Problem 4:

\section{You want to work for Google after college. You send an email to Google’s HR director and they reply ``you will be hired only if you major in mathematics or computer science, get a B average or better, and take accounting.'' You become a math major, get a B+ average, and also take some accounting courses. You return to Google, apply for a job, and are turned down. Did the HR director lie to you?}
\hspace{1cm}\textit{Logically speaking, the language ``you WILL be hired only if'' implies that, by meeting whatever criteria follow, you WILL be hired. The criteria were majoring in math OR computer science (majored in math, true), B+ average OR better (B+, true), and take accounting (took some accounting courses, assuming this satisfies ``taking accounting'', true). Therefore, since each criteria is met, and you were not hired, the statement from the HR director was false.}

\clearpage

% Problem 5:


\section{For a given predicate $P(x)$, you might believe that the statements $\forall xP(x)$ or $\exists xP(x)$ are either true or false. How would you decide if you were correct in each case? You have four choices: you could give an example of an element n in the domain for which $P(n)$ is true or for which $P(n)$ if false, or you could argue that no matter what n is, $P(n)$ is true or is false.}
\subsection{What would you need to do to prove $\forall xP(x)$ is true?}
\hspace{1cm}\textit{In order to prove $\forall xP(x)$ is true, you would need to prove that $P(x)$ remains true universally. Suppose $P(x) = 1$, and is posited to always be greater than 0. A result less than or equal to 0 would make prove the statement false. Since no such example can be found, $\forall xP(x)$ is true.}
\subsection{What would you need to do to prove $\forall xP(x)$ is false?}
\hspace{1cm}\textit{In order to prove $\forall xP(x)$ is true, you would only need to find a single example of falsehood. Suppose $P(x) = x$ and is posited to never be equal to 0. Finding a single $n$ where $P(n) = 0$ would disprove the initial assertion. If we use $n = 0$, we find a single example where $P(n) = P(0) = 0$. Therefore, the statement is not universally true, and thus, is false.}
\subsection{What would you need to do to prove $\exists xP(x)$ is true?}
\hspace{1cm}\textit{In order to prove $\exists xP(x)$ is true, you would need to find single example that $P(x)$ is true. Let's look at the previous example. Suppose $P(x) = x$ and is posited to be equal to 0. Since we can find one $n = 0, P(n) = P(0) = 0$, the statement is true.}
\subsection{What would you need to do to prove $\exists xP(x)$ is false?}
\hspace{1cm}\textit{In order to prove $\exists xP(x)$ is false, you would need to prove that $P(x)$ is universally false. For example, assume $P(x) = 1$, and is posited to equal zero. By observation, we can see that for all $x$, $P(x) \neq 0$. Therefore, the statement is proven false.}


% End document:
\end{document}
